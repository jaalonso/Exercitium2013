% Exercitium2013.tex
% Exercitium (curso 2013-14)
% José A. Alonso Jiménez <jalonso@us.es>
% Sevilla, 11 de diciembre de 2018
% ======================================================================

\documentclass[a4paper,12pt,twoside]{book}

%%%%%%%%%%%%%%%%%%%%%%%%%%%%%%%%%%%%%%%%%%%%%%%%%%%%%%%%%%%%%%%%%%%%%%%%
%% § Paquetes adicionales
%%%%%%%%%%%%%%%%%%%%%%%%%%%%%%%%%%%%%%%%%%%%%%%%%%%%%%%%%%%%%%%%%%%%%%%%

% Configuración para XeLaTeX
\usepackage{fontspec}
\usepackage{xltxtra}
\defaultfontfeatures{Ligatures=TeX,Numbers=OldStyle}
\setromanfont{DejaVu Sans}
% \setsansfont{Arial}
\setmonofont{DejaVu Sans Mono}[Scale={0.90}]

% Notas: La lista de fuentes disponibles se obtiene con fc-list

% \usepackage{ucs}
% \usepackage[utf8]{inputenc}        % Acentos de UTF8
\usepackage[spanish]{babel}        % Castellanización.
% \usepackage[T1]{fontenc}           % Codificación T1 con European Computer
%                                    % Modern.  
% \usepackage{graphicx}
% \usepackage{fancyvrb}              % Verbatim extendido
% \usepackage{mathpazo}              % Fuentes semejante a palatino
% \usepackage[scaled=.90]{helvet}
% \usepackage{cmtt}
% \renewcommand{\ttdefault}{cmtt}
\usepackage{a4wide}
\usepackage{minted}
\usepackage{comment}

\usepackage{titletoc}
\dottedcontents{chapter}[0em]{}{32em}{1pc}
% \dottedcontents{chapter}[<left>]{<above-code>}{<label width>}{<leader width>}

\linespread{1.05}                  % Distancia entre líneas
\setlength{\parindent}{2em}        % Indentación de comienzo de párrafo
\setlength{\parskip}{1ex}          % Distancia entre párrafos
% \deactivatetilden                  % Elima uso de ~ para la eñe
\raggedbottom                      % No ajusta los espacios verticales.

\usepackage[%
  colorlinks=true,
  urlcolor=blue,
  % pdftex,
  pdfauthor={José A. Alonso <jalonso@us.es>},%
  pdftitle={Exercitium (curso 2013-14)},%
  pdfstartview=FitH,%
  bookmarks=false]{hyperref}      

\setcounter{tocdepth}{1}
\setcounter{secnumdepth}{4}

\usepackage{tocstyle}
\usetocstyle{KOMAlike}

% \usepackage{tocloft}
% \renewcommand\cftpartnumwidth{3cm}

% \usepackage{minitoc}

% \setlength\cftparskip{-2pt}
% \setlength\cftbeforechapskip{0pt}

%%%%%%%%%%%%%%%%%%%%%%%%%%%%%%%%%%%%%%%%%%%%%%%%%%%%%%%%%%%%%%%%%%%%%%%%%%%%%%
%% § Cabeceras                                                              %%
%%%%%%%%%%%%%%%%%%%%%%%%%%%%%%%%%%%%%%%%%%%%%%%%%%%%%%%%%%%%%%%%%%%%%%%%%%%%%%

\usepackage{fancyhdr}

\addtolength{\headheight}{\baselineskip}

\pagestyle{fancy}

\cfoot{}

\fancyhead{}
\fancyhead[RE]{\mdseries\sffamily Exercitium (2013--14)}
\fancyhead[LO]{\mdseries\sffamily \nouppercase{\leftmark}}
\fancyhead[LE,RO]{\mdseries\sffamily \thepage}

%%%%%%%%%%%%%%%%%%%%%%%%%%%%%%%%%%%%%%%%%%%%%%%%%%%%%%%%%%%%%%%%%%%%%%%%
%% § Definiciones
%%%%%%%%%%%%%%%%%%%%%%%%%%%%%%%%%%%%%%%%%%%%%%%%%%%%%%%%%%%%%%%%%%%%%%%%

\input definiciones
\def\mtctitle{Contenido}

%%%%%%%%%%%%%%%%%%%%%%%%%%%%%%%%%%%%%%%%%%%%%%%%%%%%%%%%%%%%%%%%%%%%%%%%
%% § Título
%%%%%%%%%%%%%%%%%%%%%%%%%%%%%%%%%%%%%%%%%%%%%%%%%%%%%%%%%%%%%%%%%%%%%%%%

\title{
  {\LARGE Exercitium \\
    {\Large Ejercicios de programación funcional con Haskell \\
      {\normalsize Volumen 1: curso 2013--14
    }}}}
\author{\href{http://www.cs.us.es/~jalonso}
        {\Large José A. Alonso Jiménez}}
\date{\vfill \hrule \vspace*{2mm}
  \begin{tabular}{l}
      \href{http://www.cs.us.es/glc}
           {Grupo de Lógica Computacional} \\
      \href{http://www.cs.us.es}
           {Dpto. de Ciencias de la Computación e Inteligencia Artificial} \\
      \href{http://www.us.es}
           {Universidad de Sevilla}  \\
      Sevilla, 11 de diciembre de 2018
  \end{tabular}\hfill\mbox{}}

%%%%%%%%%%%%%%%%%%%%%%%%%%%%%%%%%%%%%%%%%%%%%%%%%%%%%%%%%%%%%%%%%%%%%%%%
%% § Documento
%%%%%%%%%%%%%%%%%%%%%%%%%%%%%%%%%%%%%%%%%%%%%%%%%%%%%%%%%%%%%%%%%%%%%%%%

% \includeonly{recusion_sobre_numeros_naturales}

% \includexmp{licencia}

\begin{document}
% \dominitoc

\maketitle
\newpage

\input{licenciaCC}
\newpage

\newpage

\mbox{} \vspace*{2cm}
\begin{flushright}
\textit{Para Guiomar}
\end{flushright}

\newpage

\tableofcontents
\clearpage

\renewcommand{\chaptername}{Ejercicio}

\chapter*{Introducción}

% \mbox{} \hspace*{1cm} 

\begin{quote}
  ``\textit{The chief goal of my work as an educator and author is to
  help people learn to write beautiful programs.}''

  (Donald Knuth en
  \href{http://www.paulgraham.com/knuth.html}{Computer programming as an art})
\end{quote}

\vspace* {1cm}

Este libro es una recopilación de las soluciones de los ejercicios
propuestos en el blog
\href{https://www.glc.us.es/~jalonso/exercitium}
     {Exercitium}\
     \footnote{\url{https://www.glc.us.es/~jalonso/exercitium}}
durante el curso 2013--14.

El principal objetivo de Exercitium es servir de complemento a la
asignatura de
\href{https://www.cs.us.es/~jalonso/cursos/i1m-13}
     {Informática}\
     \footnote{\url{https://www.cs.us.es/~jalonso/cursos/i1m-13}}
de 1º del Grado en Matemáticas de la Universidad de Sevilla.

Con los problemas de Exercitium, a diferencias de los de las
\href{https://www.cs.us.es/~jalonso/cursos/i1m-13/ejercicios/ejercicios-I1M-2013.pdf}
     {relaciones}\
     \footnote{\url{https://www.cs.us.es/~jalonso/cursos/i1m-13/ejercicios/ejercicios-I1M-2013.pdf}},
se pretende practicar con los conocimientos adquiridos durante todo el
curso, mientras que con las relaciones están orientadas a los nuevos
conocimientos.

Habitualmente de cada ejercicio se muestra distintas soluciones y se
compara sus eficiencias.

La dinámica del blog es la siguiente: cada día, de lunes a viernes, se
propone un ejercicio para que los alumnos escriban distintas soluciones
en los comentarios. Pasado 7 días de la propuesta de cada ejercicio, se
cierra los comentarios y se publica una selección de sus soluciones.

Para conocer la cronología de los temas explicados se puede consultar el
\href{https://www.glc.us.es/~jalonso/vestigium/category/curso/i1m/i1m2013}
     {diario de clase}\
     \footnote{\url{https://www.glc.us.es/~jalonso/vestigium/category/curso/i1m/i1m2013}}.

El código del libro se encuentra en
\href{https://github.com/jaalonso/Exercitium2018}
     {GitHub}\
     \footnote{\url{https://github.com/jaalonso/Exercitium2013}}

\chapter{Iguales al siguiente}
\entrada{Iguales_al_siguiente}

\chapter{Ordenación por el máximo}
\entrada{Ordenados_por_maximo}

\chapter{La bandera tricolor}
\entrada{Bandera_tricolor}

\chapter{Elementos minimales}
\entrada{ElementosMinimales}

\chapter{Mastermind}
\entrada{Mastermind}

\chapter{Primos consecutivos con media capicúa}
\entrada{Primos_consecutivos_con_media_capicua}

\chapter{Anagramas}
\entrada{Anagramas}

\chapter{Primos equidistantes}
\entrada{Primos_equidistantes}

\chapter{Suma si todos los valores son justos}
\entrada{Suma_si_todos_justos}

\chapter{Matrices de Toeplitz}
\entrada{Matriz_Toeplitz}

\chapter{Máximos locales}
\entrada{MaximosLocales}

\chapter{Lista cuadrada}
\entrada{Lista_cuadrada}

\chapter{Segmentos de elementos consecutivos}
\entrada{Segmentos_consecutivos}

\chapter{Valor de un polinomio mediante vectores}
\entrada{Valor_de_un_polinomio}

\chapter{Ramas de un árbol}
\entrada{Ramas_de_un_arbol}

% --------------------------------------------------------------------

\chapter{Alfabeto comenzado en un carácter}
\entrada{Alfabeto_desde}

% \chapter{Numeración de ternas de naturales}
% \entrada{Numeracion_de_ternas}
% 
% \chapter{Ordenación de estructuras}
% \entrada{Ordenacion_de_estructuras}
% 
% \chapter{Emparejamiento binario}
% \entrada{Emparejamiento_binario}
% 
% \chapter{Ampliación de columnas de una matriz}
% \entrada{Amplia_columnas}

% --------------------------------------------------------------------

% \chapter{Regiones en el plano}
% \entrada{Regiones}
% 
% \chapter{Elemento más repetido}
% \entrada{Mas_repetido}
% 
% \chapter{Número de pares de elementos adyacentes iguales}
% \entrada{Pares_adyacentes_iguales}
% 
% \chapter{Mayor producto de las ramas de un árbol}
% \entrada{Mayor_producto_de_las_ramas_de_un_arbol}
% 
% \chapter{Biparticiones de una lista}
% \entrada{Biparticiones_de_una_lista}
% 
% \chapter{Trenzado de listas}
% \entrada{Trenza}
% 
% \chapter{Números triangulares con n cifras distintas}
% \entrada{Triangulares_con_cifras}
% 
% \chapter{Enumeración de árboles binarios}
% \entrada{Enumera_arbol}
% 
% \chapter{Elementos con algún vecino menor}
% \entrada{Algun_vecino_menor}
% 
% \chapter{Reiteración de una función}
% \entrada{Reiteracion_de_funciones}
%     
% \chapter{Pim, Pam, Pum y divisibilidad}
% \entrada{PimPamPum}
% 
% \chapter{Código de las alergias}
% \entrada{Alergias}
% 
% \chapter{Índices de valores verdaderos}
% \entrada{Indices_verdaderos}
% 
% \chapter{Descomposiciones triangulares}
% \entrada{Descomposiciones_triangulares}
% 
% \chapter{Número de inversiones}
% \entrada{Numero_de_inversiones}
% 
% \chapter{Sepación por posición}
% \entrada{Separacion_por_posicion}
% 
% \chapter{Emparejamiento de árboles}
% \entrada{EmparejamientoDeArboles}
% 
% \chapter{Eliminación de las ocurrencias unitarias}
% \entrada{Elimina_unitarias}
% 
% \chapter{Ordenada cíclicamente}
% \entrada{Ordenada_ciclicamente}
% 
% \chapter{Órbita prima}
% \entrada{Orbita_prima}
% 
% \chapter{Divisores de un número con final dado}
% \entrada{Divisores_con_final}
% 
% \chapter{Descomposiciones como sumas de n sumandos}
% \entrada{Descomposiciones_con_n_sumandos}
% 
% \chapter{Selección hasta el primero que falla inclusive}
% \entrada{Seleccion_con_fallo}
% 
% \chapter{Buscaminas}
% \entrada{Buscaminas}
% 
% \chapter{Mayor sucesión del problema 3n+1}
% \entrada{Mayor_sucesion_3n%2B1}
% 
% \chapter{Filtro booleano}
% \entrada{Filtro_booleano}
% 
% \chapter{Entero positivo de la cadena}
% \entrada{Entero_positivo_de_la_cadena}
% 
% \chapter{N gramas}
% \entrada{n_gramas}
% 
% \chapter{Sopa de letras}
% \entrada{Sopa_de_letras}
% 
% \chapter{Intercalación de n copias}
% \entrada{Intercala_n_copias}
% 
% \chapter{Eliminación de n elementos}
% \entrada{Elimina_n_elementos}
% 
% \chapter{Límite}
% \entrada{Limite}
% 
% \chapter{Número de palabas que empiezan por mayúscula}
% \entrada{Empiezan_con_mayuscula}
% 
% \chapter{Renombramiento de un árbol}
% \entrada{Renombra_arbol}
% 
% \chapter{Divide si todos son múltiplos}
% \entrada{Divide_si_todos_multiplos}
% 
% \chapter{Ventana deslizante}
% \entrada{Ventana_deslizante}
% 
% \chapter{Representación de Zeckendorf}
% \entrada{Representacion_de_Zeckendorf}
% 
% \chapter{Más cercano cumpliendo una propiedad}
% \entrada{Mas_cercano_cumpliendo_la_propiedad}
% 
% \chapter{Código Morse}
% \entrada{Codigo_Morse}
% 
% \chapter{Producto cartesiano de una familia de conjuntos}
% \entrada{Producto_cartesiano}
% 
% \chapter{Todas tienen par}
% \entrada{Todas_tienen_par}
% 
% \chapter{Sucesiones pucelanas}
% \entrada{Sucesiones_pucelanas}
% 
% \chapter{Producto de matrices como listas de listas}
% \entrada{Producto_de_matrices_como_listas_de_listas}
% 
% \chapter{Inserción en árboles binarios de búsqueda}
% \entrada{Insercion_en_arboles_binarios_de_busqueda}
% 
% \chapter{Matriz permutación}
% \entrada{Matriz_permutacion}
% 
% \chapter{Números con todos sus dígitos primos}
% \entrada{Numeros_con_digitos_primos}
% 
% \chapter{Cadenas de ceros y unos}
% \entrada{Cadenas0y1}
% 
% \chapter{Clausura de un conjunto respecto de una función}
% \entrada{Clausura}
% 
% \chapter{Sustitución en una expresión}
% \entrada{Sustitucion_en_una_expresion}
% 
% \chapter{Laberinto numérico}
% \entrada{Laberinto_numerico}

\end{document}

%%% Local Variables: 
%%% mode: latex
%%% TeX-master: t
%%% End: 

